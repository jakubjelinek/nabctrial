\documentclass[a4paper]{article}
\usepackage{color}
\usepackage{luacode}
\usepackage[margin=2cm]{geometry}
\usepackage{fontspec}
\usepackage{multicol}
\usepackage{fixltx2e}
\usepackage{hyperref}
\usepackage{color}
\setlength{\columnsep}{0.3cm} \setlength{\columnseprule}{1pt}
\setmainfont{Linux Libertine O}
\pagestyle{empty}
\newfontface\GreGall{gregall.ttf}
\newfontface\GreGallModern{SGModern.ttf}
\directlua{dofile('gregall.lua')}
\newcommand{\gregallcharno}[2]{{\directlua{
  tex.sprint(gregallparse.neumes("\luaescapestring{#1}", "\luaescapestring{#2}"))
}}}
\def\gregallchar{%
  \begingroup %
    \catcode`\~=12{}%
    \fontsize{8}{8}%
    \color{red}%
    \dogregallchar%
}
\def\dogregallchar#1{
    \gregallcharno{#1}{gregall}%
  \endgroup %
}
\def\gregallmodchar{%
  \begingroup %
    \catcode`\~=12{}%
    \fontsize{16}{16}%
    \color{red}%
    \dogregallmodchar%
}
\def\dogregallmodchar#1{
    \gregallcharno{#1}{gregallmod}%
  \endgroup %
}

\begin{document}
I'm working on free fonts of St. Gall neumes for the Gregorio project.
So far there are two fonts, SGModern, which is mostly based on the work
from the Bellaigues abbey (about 200 neumatic glyphs) and I've just added
further 200 neumatic glyphs, and Gregall, where I'm trying to use exact
shapes from the manuscripts (Cantatorium, Hartker antiphonary, Einsiedeln
121, Sg. Gall 339 and Bamberg lit. 6).  I'd like to have first complete
coverage of Cardine's Table of Neumatic Signs of St. Gall and if that is
done, complete most common neumatic elements which can't be easily combined
from other neumatic elements plus prepunctis and subpunctis.
The SGModern font right now doesn't have some significative letters, the
Gregall font is missing various neumatic glyphs.
Both fonts are temporarily available from
\url{http://people.redhat.com/jakub/gregall/}.

Any thoughts where to find the following neumatic
elements (manuscript, page and perhaps line) in the manuscripts?  From the Cardine's Table
I'm missing climacus with tenete on it
{\color{red}{\textsuperscript{t}}\gregallmodchar{ci}}\quad, scandicus with episemas
\gregallmodchar{sc-1}\quad, melodic scandicus form \gregallmodchar{scM}\quad,
pes with one subpunctum and one stropha \gregallmodchar{pesu1sux1}\quad,
2nd form of scandicus flexus with neumatic break \gregallmodchar{sfG1}\quad,
melodic form of scandicus flexus \gregallmodchar{sfM1}\quad, liquescens scandicus
flexus \gregallmodchar{sf>}\quad (only have one with two tractulis instead of punctis
\gregallchar{cl>ppt2}\quad), torculus resuspinus with celeriter on it
{\color{red}{\textsuperscript{c}}\gregallmodchar{tr}}\quad, torculus resuspinus with
episema \gregallmodchar{tr-}\quad, one modified shape form of trigonus
\gregallmodchar{tgS2}\quad, pressus minor with gravis
\gregallmodchar{piM}\quad, pressus maior with stropha \gregallmodchar{pr>}\quad, salicus with
frendor on it {\color{red}{\textsuperscript{f}}\gregallmodchar{saM1}}\quad and
one melodic form of salicus \gregallmodchar{saM}\quad.  Anyone knows where in the MSS
to find significative letters {\color{red}pulcre}?

In the Cardine's table, I'm seeing various glyphs used for two or more
neumatic signs, is that correct or just I'm not able to figure out the
differences in between them?  In particular, is the ancus glyph
\gregallmodchar{ci~}\quad\gregallchar{ci~} the same glyph for 1st augment. clivis and
diminut. climacus, or is there any difference in between those two?
Similarly is the epiphonus glyph \gregallmodchar{pe~}\quad\gregallchar{pe~}\quad the same
glyph for augment. tractulus, diminut. pes and diminut. virga strata?
Is augment. virga strata \gregallmodchar{vs>}\quad\gregallchar{vs>}\quad the same
glyph as diminut. pressus maior?  Is 3rd augment. form of pes
\gregallmodchar{pe>2}\quad\gregallchar{pe>2}\quad the same glyph as diminut. scandicus
and diminut. salicus (tractulus + epiphonus)?  What about augment. porrectus
\gregallmodchar{po>}\quad\gregallchar{po>}\quad and diminut. porrectus flexus?
Augment. torculus \gregallmodchar{to>}\quad\gregallchar{to>}\quad and diminut. pes
subbipunctis?  Augment. virga (cephalicus) \gregallmodchar{vi>}\quad\gregallchar{vi>}\quad
and diminut. clivis?  Porrectus flexus with neumatic break
\gregallmodchar{sfM}\quad\gregallchar{sfM}\quad and 1st melodic form
of scandicus flexus?

As the majority of the SGModern font was designed by others, I'd appreciate
feedback on matching of some of the glyphs in that font with what I found in
the manuscripts and what is in the Gregorian Semiology.  Most importantly,
the diminut. porrectus \gregallchar{po~}\quad, diminut. torculus resuspinus
\gregallchar{tr~}\quad and the same with episema \gregallchar{tr-~}\quad look to me like virga or
pes with epiphonus attached to the top, while the SGModern font have
glyphs with the epiphonus attached to the bottom \gregallmodchar{po~}\quad,
\gregallmodchar{tr~}\quad, \gregallmodchar{tr-~}\quad, which I haven't found in the
manuscripts in such a form.  Can that be considered the same glyphs?
Also, in the manuscripts I also found (in many places) a form with quadratic
form of the beginning of the torculus \gregallchar{tr-~1}\quad, is that an
important glyph difference or just a small detail that can be ignored?

The second augment. form of clivis \gregallchar{cl>1} has the right lower curve much
bigger than the glyphs in SGModern \gregallmodchar{cl>1} (and similarly with episema),
is that also something that can be easily understood as the same neumatic
element?

For some of the shape modifications of torculus and torculus resuspinus, I'm
afraid I don't clearly grok which characters are which.  For the 2nd shape
modified torculus in the Cardine's table, the SGModern font has two quite
similar glyphs \gregallmodchar{toS1}\quad and \gregallmodchar{toS4}\quad, what I found in the
manuscripts is only \gregallchar{toS1}\quad, is that the first or second SGModern
glyph and which of those two is the one in Cardine's table?
Similarly the 2nd shape modified torculus resuspinus, SGModern has
\gregallmodchar{trS1}\quad, what I found in the MSS is \gregallchar{trS1}\quad, but not sure
if it is the same neume.  For the 3rd shape modified torculus resuspinus,
I'm not sure what is the difference from the plain torculus resuspinus
\gregallmodchar{tr}\quad\gregallchar{tr}\quad.

Also, I'd like to ask, some glyphs like clivis combined with pressus come
in two different forms:
\gregallmodchar{cl!pi}\quad\gregallchar{cl!pi}\quad and
\gregallmodchar{cl!pr}\quad\gregallchar{cl!pr}\quad.  Is the first one clivis with
pressus minor and the second one with pressus maior, or is that both pressus
minor, just in different shape?

In the Einsiedeln 121 codex, I've often found a salicus where the shape of
the punctum looked much more like a tractulus than punctum, but it seems
in Graduale Triplex that is usually painted just as punctum.  Is that just
an insignificant change one can ignore?

The SGModern font has separate {\fontsize{16}{16}\color{red}\GreGallModern\char200}\quad and \gregallmodchar{pe>1}\quad
characters, all I found in the MSS so far was \gregallchar{pe>1}\quad, is that
the first or second glyph in SGModern?  Which one is the 2nd augment. pes form in
Cardine's table, and how can be the other neume classified?

For all the glyphs in the Gregall font, I have notes from where I've picked
those up, have a 4 month old document at my web pages which describes that,
but as about 100 glyphs has been added to that font since then, it is out of
date, if anyone wants to know a particular glyph, I can answer it quickly,
otherwise I'll try to eventually update that document, but it might take
some time.

\vfill
\leavevmode\gregallchar{ql-ppt1suu2}
\vfill
\leavevmode\gregallchar{ql-}
\vfill
\leavevmode\gregallchar{ql-ppt1}
\vfill
\leavevmode\gregallchar{ql-suu2}
\vfill
\leavevmode\gregallchar{qi!vsppt1su2}
\vfill
\leavevmode\gregallchar{qi!vsppt1}
\vfill
\leavevmode\gregallchar{qi!vssu2}
\vfill

\begin{luacode*}
  local i
  local last
  local key
  local str
  local keym
  local arr = {}
  for key in pairs(gregalltab['gregall']) do
    table.insert(arr, key)
  end
  for key in pairs(gregalltab['gregallmod']) do
    table.insert(arr, key)
  end
  table.sort(arr)
  tex.sprint("\\begin{multicols}{3}\\noindent");
  for i, key in ipairs(arr) do
    if last ~= key then
      last = key
      keym = key:gsub("~","\\~{}")
      str = keym .. "\\hfill"
      if gregalltab['gregall'][key] then
        str = str .. "{\\fontsize{8}{8}\\color{red}" .. gregalltab['gregall'][key] .. "}"
      else
        str = str .. "{\\color{blue}missing}"
      end
      str = str .. "\\hfill"
      if gregalltab['gregallmod'][key] then
        str = str .. "{\\fontsize{16}{16}\\color{red}" .. gregalltab['gregallmod'][key] .. "}"
      else
        str = str .. "{\\color{blue}missing}"
      end
      tex.sprint(str .. "\\\\")
    end
  end
  tex.sprint("\\end{multicols}")
\end{luacode*}
\end{document}
