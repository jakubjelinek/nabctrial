% !TEX TS-program = lualatex
% !TEX encoding = UTF-8

% This is a simple template for a LuaLaTeX document using gregorio scores.

\documentclass[12pt]{article} % use larger type; default would be 10pt

% usual packages loading:
\usepackage{fontspec}
\usepackage{graphicx} % support the \includegraphics command and options
%\usepackage{geometry} % See geometry.pdf to learn the layout options. There are lots.
%\geometry{a5paper} % or letterpaper (US) or a5paper or....
\usepackage[a5paper,left=0.2cm,right=0.5cm,top=0.2cm,bottom=0.2cm]{geometry}
\usepackage{gregoriotex} % for gregorio score inclusion
\usepackage[cm]{fullpage} % to reduce the margins
%\pagenumbering{gobble}% Remove page numbers (and reset to 1)
\pagestyle{empty}

\textwidth 5in

\newfontface\GreGall{gregall.ttf}
\newfontface\GreGallModern{SGModern.ttf}
\directlua{dofile('gregall.lua')}
\newcommand{\gregallcharno}[3]{{\directlua{
  tex.sprint(gregallparse_neumes("\luaescapestring{#1}", "\luaescapestring{#2}", \luaescapestring{#3}))
}}}
\def\gregallchar{%
  \begingroup %
    \catcode`\~=12{}%
    \catcode`\@=11{}%
    \fontsize{8}{8}%
    \color{red}%
    \dogregallchar%
}
\def\dogregallchar#1{%
    \gregallcharno{#1}{gregall}{0.8}%
  \endgroup %
}
\def\gregallmodchar{%
  \begingroup %
    \catcode`\~=12{}%
    \catcode`\@=11{}%
    \fontsize{16}{16}%
    \color{red}%
    \dogregallmodchar%
}
\def\dogregallmodchar#1{%
    \gregallcharno{#1}{gregallmod}{1.6}%
  \endgroup %
}


% to change the font to something better, you can install the kpfonts package (if not already installed). To do so
% go open the "TeX Live Manager" in the Menu Start->All Programs->TeX Live 2010

% here we begin the document
\begin{document}

% Here we set the space around the initial.
% Please report to http://home.gna.org/gregorio/gregoriotex/details for more details and options
\setspaceafterinitial{1mm plus 0em minus 0em}
\setspacebeforeinitial{1mm plus 0em minus 0em}

% Here we set the initial font. Change 43 if you want a bigger initial.
\def\greinitialformat#1{%
{{\color{red}\fontsize{46}{46}\selectfont #1}}%
}

% We set IV above the initial.
%\gresetfirstlineaboveinitial{\tiny \textsc{\textbf{II D}}}{\tiny \textsc{\textbf{II D}}}

% We type a text in the top right corner of the score:
%\commentary{{\small \emph{XVII. s.}}}

% and finally we include the score. The file must be in the same directory as this one.
\includescore{ant1.tex}
\vfill

\includescore{ant2.tex}
\vfill

\includescore{ant3.tex}
\vfill

\includescore{ant4.tex}
\vfill

\includescore{ant5.tex}
\vfill

\includescore{ant-magn-vesp1.tex}
\vfill

\includescore{ant-magn-vesp2.tex}
\vfill

\includescore{ant-ben-laud.tex}
\vfill

\includescore{matant1.tex}
\vfill

\includescore{matant2.tex}
\vfill

\includescore{matant3.tex}
\vfill

\includescore{matant4.tex}
\vfill

\includescore{matant5.tex}
\vfill

\includescore{matant6.tex}
\vfill

\includescore{matant7.tex}
\vfill

\includescore{matant8.tex}
\vfill

\includescore{matant9.tex}
\vfill

\includescore{matresp1.tex}
\vfill

\includescore{matresp2.tex}
\vfill

\includescore{matresp4.tex}
\vfill

\includescore{matresp5.tex}
\vfill

\includescore{matresp7.tex}
\vfill

\end{document}
